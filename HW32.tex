\documentclass[12pt]{amsart}
\usepackage[margin=1in]{geometry}
\usepackage{ amssymb }

\newcommand{\Z}{\mathbb{Z}}
\newcommand{\N}{\mathbb{N}}
\renewcommand{\complement}{^{\mathsf{c}}}

\newtheorem*{proposition}{Proposition}

\title{301 Homework 32}
\author{Mark Kim}
\date{\today}

\begin{document}
\maketitle

\begin{proposition}
The function $f\colon \N\rightarrow\Z$ defined by
\begin{align*}
f(n) =
\begin{cases}
\frac{1-n}{2},\text{\hspace{.1cm}when $n$ is odd}\\
\frac{n}{2},\text{\hspace{.5cm}when $n$ is even}
\end{cases}
\end{align*}
is bijective.
\end{proposition}

\begin{proof}
Assume $f\colon \N\rightarrow\Z$ defined by
\begin{align*}
f(n) =
\begin{cases}
\frac{1-n}{2},\text{\hspace{.1cm}when $n$ is odd}\\
\frac{n}{2},\text{\hspace{.5cm}when $n$ is even}
\end{cases}
\end{align*}
is bijective.

Define $g\colon\Z\rightarrow\N$ as
\begin{align*}
g(m) =
\begin{cases}
1-2m,\text{\hspace{.1cm}}m\leq0\\
2m,\text{\hspace{.5cm}}m>0.
\end{cases}
\end{align*}

First we need to examine the left inverse, which we need to further break up into odd and even cases.  Notice that when n is odd,
\begin{align*}
g(f(n)) = g\left(\frac{1-n}{2}\right) = 1-2\left(\frac{1-n}{2}\right) = n.
\end{align*}
Similarly, when n is even,
\begin{align*}
g(f(n)) = g\left(\frac{n}{2}\right) = 2\left(\frac{n}{2}\right) = n.
\end{align*}

We can examine the right inverse in a similar manner by looking at cases where $m\leq0$ and $m>0$.  Consider that when $m\leq0$,
\begin{align*}
f(g(m)) = f\left(1-2m\right) = \frac{1-(1-2m)}{2} = m.
\end{align*}
When $m>0$,
\begin{align*}
f(g(m)) = f\left(2m\right) = 2\left(\frac{n}{2}\right) = m.
\end{align*}

Note that it is never the case where $n$ is odd and $m>0$.  Furthermore, it is also never the case where $n$ is even and $m\leq0$.  Since $g$ is the left and right inverse of $f$, $f$ is bijective.
\end{proof}
\end{document}