\documentclass[12pt]{amsart}
\usepackage[margin=1in]{geometry}

\newcommand{\Z}{\mathbb{Z}}
\newcommand{\N}{\mathbb{N}}

\newtheorem*{proposition}{Proposition}

\title{301 Homework 12}
\author{Mark Kim}
\date{\today}

\begin{document}
\maketitle

\begin{proposition}
For all $k\in\N$, $k^3+5k$ is divisible by 6.
\end{proposition}

\begin{proof}
Let $k\in\N$. We proceed by induction and must prove the base case and the induction step.  Using the base case, $k=1$, notice that $1^3+5\cdot1=6$ is divisible by $6$.  Continuing with the induction step, we assume $m^3+5m=6i$ for some $m\in\N$ and $i\in\Z$.  Observe that
\begin{align*}
(m+1)^3 + 5(m+1) &= m^3 + 3m^2 + 3m + 1 + 5m + 1\\
&= (m^3 + 5m) + (3m^2 + 3m +6)\\
&= 6i + (3m^2 + 3m + 6).
\end{align*}
To progress, we repeat induction for $3m^2 + 3m + 6$.  Using the base case $j=1$, it can be surmised that $3\cdot1^2 + 3\cdot1 + 6 = 12$ is divisible by $6$.  Moving to the induction step, we assume $3m^2 + 3m + 6 = 6j$ for some $m\in\N$ and $j\in\Z$.  We can now deduce that
\begin{align*}
3(m+1)^2 + 3(m+1) + 6 &= 3m^2+6m+3+3m+3\\
&= (3m^2 + 3m +6) + 6m\\
&= 6j + 6m.
\end{align*}
Substituting $6j + 6m$ back into $6i + (3m^2 + 3m + 6)$ yields us $6i + 6j + 6m$ which simplifies to $6(i+j+m)$, proving that for all $k\in\N$, $k^3+5k$ is divisible by 6.
\end{proof}

\end{document}