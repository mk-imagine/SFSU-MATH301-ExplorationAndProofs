\documentclass[12pt]{amsart}
\usepackage[margin=1in]{geometry}

\newcommand{\Z}{\mathbb{Z}}

\newtheorem*{proposition}{Proposition}

\title{301 Homework 3}
\author{Mark Kim}
\date{\today}

\begin{document}
\maketitle

\begin{proposition}
If $m$, $n$, $p$, and $q$ are integers, then $(m(n+p))q = (mn)q + m(pq)$.
\end{proposition}

\begin{proof}
Let $m,n,p,q\in\Z$. By Axiom 1.1(iii), we can distribute the $m$ in $m(n+p)$ within the equation $(m(n+p))q$ to form the expression $(mn+mp)q$. 
Similarly, by applying Proposition 1.6, we can distribute $q$ in $(mn+mp)q$, which results in $(mn)q+(mp)q$.  The final step requires us to use Axiom 1.1(v) to relocate the parentheses in $(mp)q$ to form $m(pq)$, which shows that $(mn)q+(mp)q$ can be expressed as $(mn)q + m(pq)$.  In summary we have proved that
\begin{align*}
(m(n+p))q&\stackrel{\text{Ax 1.1(iii)}}{=}(mn+mp)q\\
&\stackrel{\text{Prop 1.6}}{=}(mn)q+(mp)q\\
&\stackrel{\text{Ax 1.1(v)}}{=}(mn)q+m(pq).
\end{align*}
In conclusion, $(m(n+p))q = (mn)q + m(pq)$.
\end{proof}



\end{document}