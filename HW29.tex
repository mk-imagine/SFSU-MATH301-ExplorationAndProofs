\documentclass[12pt]{amsart}
\usepackage[margin=1in]{geometry}
\usepackage{ amssymb }

\newcommand{\Z}{\mathbb{Z}}
\newcommand{\N}{\mathbb{N}}
\renewcommand{\complement}{^{\mathsf{c}}}

\newtheorem*{proposition}{Proposition}

\title{301 Homework 29}
\author{Mark Kim}
\date{\today}

\begin{document}
\maketitle

\begin{proposition}
``Having the same cardinality'' is an equivalence relation on the collection of all sets.
\end{proposition}

\begin{proof}
Assume set $A$ has the same cardinality as set $B$, which, by definition, means there is a bijection $f\colon A\rightarrow B$.  To prove that having the same cardinality is an equivalence relation, we must prove reflexivity, symmetry, and transitivity.

To prove reflexivity, we define a bijection $f\colon A\rightarrow A$ such that $f(a)=a$, which proves set $A$ has the same cardinality as $A$.

Next, we define a bijection $f\colon A\rightarrow B$.  Using Proposition 9.10(iii) allows us to assert that $f$ has an inverse which we can define as $g\colon B\rightarrow A$.  By definition, because $g$ is the inverse of $f$, it is also bijective, which proves symmetry.

To prove transitivity, define a bijection $f\colon A\rightarrow B$ and bijection $g\colon B\rightarrow C$, such that $f(a) = b$ and $g(b) = c$.  Using Proposition 9.7(iii),  we can assert that the composition $g(f(a)) = c$ is also bijective.

Therefore ``Having the same cardinality'' is an equivalence relation on the collection of all sets.
\end{proof}

\end{document}