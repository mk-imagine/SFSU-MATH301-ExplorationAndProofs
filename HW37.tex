\documentclass[12pt]{amsart}
\usepackage[margin=1in]{geometry}
\usepackage{ amssymb }

\newcommand{\Z}{\mathbb{Z}}
\newcommand{\N}{\mathbb{N}}
\newcommand{\R}{\mathbb{R}}
\renewcommand{\complement}{^{\mathsf{c}}}

\newtheorem*{proposition}{Proposition}

\title{301 Homework 37}
\author{Mark Kim}
\date{\today}

\begin{document}
\maketitle

\begin{proposition}
Let $x,y\in\R$.  Then $x=y$ if and only if for every $\epsilon > 0$ we have $\lvert x-y\rvert < \epsilon$.
\end{proposition}

\begin{proof}
Let $x,y\in\R$ and $x=y$.  By Proposition 10.10(i), $\lvert x-y\rvert = 0$, which means that  $\lvert x-y\rvert = 0 < \epsilon$ for all $\epsilon > 0$.

Considering the opposite direction, to produce a contradiction, let us assume that $\lvert x-y\rvert < \epsilon$ for all $\epsilon > 0$  such that $x\neq y$.  By definition, $\lvert x-y\rvert \geq 0$, but applying Proposition 10.10(i) allows us to write $\lvert x-y\rvert > 0$.

Let $\epsilon = \frac{1}{2}\lvert x-y\rvert$, which allows us to maintain that $\epsilon > 0$.  But this means that
\begin{align*}
\lvert x-y\rvert > \epsilon = \frac{1}{2}\lvert x-y\rvert,
\end{align*}
producing a contradiction which proves our proposition.
\end{proof}
\end{document}