\documentclass[12pt]{amsart}
\usepackage[margin=1in]{geometry}
\usepackage{ amssymb }

\newcommand{\Z}{\mathbb{Z}}
\newcommand{\N}{\mathbb{N}}
\renewcommand{\complement}{^{\mathsf{c}}}

\newtheorem*{proposition}{Proposition}

\title{301 Homework 20}
\author{Mark Kim}
\date{\today}

\begin{document}
\maketitle

\begin{proposition}
Assume we are given an equivalence relation on a set $A$.  For all $a_1,a_2\in A$, $[a_1] = [a_2]$ or $[a_1] \cap [a_2] = \varnothing$.
\end{proposition}

\begin{proof}
Let $a_1,a_2\in A$.  We have two cases: $a_1 \sim a_2$ or $a_1 \not\sim a_2$.

For our first case, $a_1 \sim a_2$, invoking Proposition 6.4(ii) results in $[a_1] = [a_2]$.

For our second case, $a_1 \not\sim a_2$, let us assume $x \in [a_1]$ and $x \in [a_2]$ to produce a contradiction.  By definition, this means $x \sim a_1$ and $x \sim a_2$.  By the property of transitivity, $a_1 \sim a_2$, producing a contradiction.  Therefore we have proven that $a_1 \not\sim a_2$.




%$a_1 \in [a_1]$ and $a_2 \notin [a_1]$.  Furthermore, $a_2 \notin [a_1]$ can be represented as $a_2 \in [a_1]\complement$.  Since it can never be the case that $a \in [a_1]$ and $a \notin [a_1]$, we can write $[a_1] \cap [a_1]\complement = \varnothing$.  Because $a_2 \in [a_2]$ and $a_2 \in [a_1]\complement$, it can be deduced that $[a_1] \cap [a_2] = \varnothing$, proving our proposition.
\end{proof}

\end{document}