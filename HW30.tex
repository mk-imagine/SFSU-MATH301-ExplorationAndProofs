\documentclass[12pt]{amsart}
\usepackage[margin=1in]{geometry}
\usepackage{ amssymb }

\newcommand{\Z}{\mathbb{Z}}
\newcommand{\N}{\mathbb{N}}
\renewcommand{\complement}{^{\mathsf{c}}}

\newtheorem*{proposition}{Proposition}

\title{301 Homework 30}
\author{Mark Kim}
\date{\today}

\begin{document}
\maketitle

\begin{proposition}
If $m>n$ then a function $[m]\rightarrow [n]$ cannot be injective.
\end{proposition}

\begin{proof}
Assume $m,n\in\N$ and $m>n$. We prove the proposition by proceeding with induction on $n$.

For the base case, $n=1$, $m>1$, and $[m]\rightarrow [1]$.  The single element in $[1]$ is an image of every element in $[m]$, and since $m>1$, the function $[m]\rightarrow [1]$ is certainly not injective.

For the induction step, let $n\geq 2$ and assume that $P(n-1)$ is not injective.  Towards a contradiction, suppose there exists an injection for $P(n)$.  This means that there exists an injection $[m]\rightarrow[n]$.  Since there is an injection $[m]\rightarrow[n]$, we can say that if we remove an element from $[m]$ and its unique image $f(m)$ from $[n]$, we maintain an injection
\[
[m-1]\rightarrow[n]-\{f(m)\}.
\]
By applying Proposition 13.3, we can assert that the function
\[
[n]-\{f(m)\}\rightarrow[n-1]
\]
is bijective.  Proposition 9.7 states that since $[m-1]\rightarrow[n]-\{f(m)\}$ is injective and $[n]-\{f(m)\}\rightarrow[n-1]$ is bijective, $[m-1]\rightarrow[n-1]$ is injective, which contradicts our induction hypothesis that $P(n-1)$ is not injective.
\end{proof}

\end{document}