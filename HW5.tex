\documentclass[12pt]{amsart}
\usepackage[margin=1in]{geometry}

\newcommand{\Z}{\mathbb{Z}}

\newtheorem*{proposition}{Proposition}

\title{301 Homework 5}
\author{Mark Kim}
\date{\today}

\begin{document}
\maketitle

\begin{proposition}
If $m$ and $n$ are even integers, then so are $m+n$ and $mn$.
\end{proposition}

\begin{proof}
We begin by assuming that we are given even numbers $m,n\in\Z$.  We need to prove that $m+n$ and $mn$ are also even. First, we are provided with the definition that an even number $m$ is divisible by $2$ such that $m=i2$ where $i\in\Z$.  Likewise, we can depict $n$ as $j2$ where $j\in\Z$.  Let us first evaluate the statement $m+n$.  Using the property of replacement, we can substitute $i2$ for $m$ and $i2$ for $n$, which allows us to write the expression $m+n$ as $i2 + j2$.  By using Proposition 1.6, it follows that from $i2 + j2$, we can form $(i + j)2$.  From this, we can now deduce that $m+n=(i + j)2$.  Moving onwards to the statement $mn$,  we can also substitute $i2$ for $m$ and $j2$ for $n$ to form the statement $(i2)\cdot(j2)$.  Axiom 1.1(v) allows us to move the parentheses in $(i2)\cdot(j2)$ to form $((i2)j)2$.  Again, it can be determined that $mn=((i2)j)2$.  Thus we have poven that
\begin{align*}
m+n&\stackrel{\text{Substitute}}{=}i2 + j2\\
&\stackrel{\text{Ax 1.1(iii)}}{=}(i + j)2.
\end{align*}
Furthermore, we have also poven that
\begin{align*}
mn&\stackrel{\text{Substitute}}{=}(i2) \cdot (j2)\\
&\stackrel{\text{Ax 1.1(v)}}{=}((i2)j)2.
\end{align*}
In conclusion, if $m$ and $n$ are even integers, then so are $m+n$ and $mn$.
\end{proof}



\end{document}