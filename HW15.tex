\documentclass[12pt]{amsart}
\usepackage[margin=1in]{geometry}
\usepackage{amsmath}
\newcommand{\Z}{\mathbb{Z}}
\newcommand{\N}{\mathbb{N}}

\newtheorem*{proposition}{Proposition}

\title{301 Homework 15}
\author{Mark Kim}
\date{\today}

\begin{document}
\maketitle

\begin{proposition}
Let $k\in\N$.
\[
\sum_{j=1}^{k}j^2 = \frac{k(k+1)(2k+1)}{6}.
\]
\end{proposition}

\begin{proof}
We proceed by induction. Using the base case, $k=1$, notice that
\[
\sum_{j=1}^{1}1^2 = \frac{1(1+1)(2\cdot 1+1)}{6} = \frac{6}{6} = 1.
\]
Continuing with the induction step we assume
\[
\sum_{j=1}^{n}j^2 = \frac{n(n+1)(2n+1)}{6}
\]
for some $k\in\N$.  Through the definition of summation, we can write
\[
\sum_{j=1}^{n + 1}j^2 = \left(\sum_{j=1}^{n}j^2\right) + (n+1)^2.
\]
We can now observe the following:
\begin{align*}
&= \frac{n(n+1)(2n+1)}{6} + (n+1)^2\\
&= \frac{n(n+1)(2n+1) + 6(n+1)^2}{6}\\
&= \frac{(n+1)(n(2n+1) + 6(n+1))}{6}\\
&= \frac{(n+1)(2n^2 + 7n + 6)}{6}\\
&= \frac{(n+1)(n + 2)(2n + 3)}{6}\\
&= \frac{(n+1)((n + 1) + 1)(2(n + 1) + 1)}{6}.
\end{align*}
This completes our induction step, proving the proposition.
\end{proof}

\end{document}