\documentclass[12pt]{amsart}
\usepackage[margin=1in]{geometry}
\usepackage{ amssymb }

\newcommand{\Z}{\mathbb{Z}}
\newcommand{\N}{\mathbb{N}}
\renewcommand{\complement}{^{\mathsf{c}}}

\newtheorem*{proposition}{Proposition}

\title{301 Homework 22}
\author{Mark Kim}
\date{\today}

\begin{document}
\maketitle

\begin{proposition}
If $a\equiv a'$ (mod $n$) and $b\equiv b'$ (mod $n$) then
\[
\begin{tabular}{ c c c c c }
$a+b\equiv a' + b'$  (mod $n$) & \hspace{0.75cm} & and & \hspace{0.75cm} & $ab \equiv a'b'$  (mod $n$).
\end{tabular}
\]
\end{proposition}

\begin{proof}
We need to check if our proposition satisfies reflexivity, symmetry, and reflexivity.  $(a+b)\equiv (a+b)$ means $n$ divides $(a+b) - (a+b) = 0$, and $ab\equiv ab$ means $n$ divides $ab-ab=0$; both statements are true.

Next we assume $a\equiv a'$ (mod $n$) and $b\equiv b'$ (mod $n$).  This can be expressed as follows:
\[
\begin{tabular}{ c c c c c }
$n|(a-a')$ & \hspace{1cm} &  & \hspace{1cm} & $n|(b-b')$\\
$a-a' = in$ & \hspace{1cm} & & \hspace{1cm} & $b-b' = jn$\\
$a = in + a'$ & \hspace{1cm} &  & \hspace{1cm} & $b = jn + b'$.
\end{tabular}
\]
Notice that
\begin{align*}
a+b &= (in +a') + (jn + b')\\
a+b &= (in + jn) + (a' + b')\\
(a+b) - (a' + b') &= (i+j)n,
\end{align*}
which means $a+b\equiv a' + b'$.
But then $(a'+b') - (a + b) = -(i+j)n$ is also divisible by $n$, so $a'+b' \equiv a+b$.

Assume $a+b\equiv a' + b'$ and $a' + b' \equiv c+d$ which means that there exists integers $\tilde{i}$ and $\tilde{j}$ such that
\[
\begin{tabular}{ c c c c c }
$(a+b) - (a'-b') = \tilde{i}n$ & \hspace{1cm} & and & \hspace{1cm} & $(a'+b') - (c-d) = \tilde{j}n$.
\end{tabular}
\]
Because of this
\[
(a+b)-(c+d) = ((a+b) - (a' + b')) + ((a' + b')-(c+d)) = \tilde{i}n + \tilde{j}n = (\tilde{i} + \tilde{j})n
\]
which means $(a+b) \equiv (c+d)$.

Similarly, we can observe that
\begin{align*}
ab &= a(jn + b')\\
ab &= ajn + ab'\\
ab &= (in + a')jn + (in + a')b'\\
ab &= injn + a'jn + b'in + a'b'\\
ab - a'b' &= (ijn + a'j + b'i)n,
\end{align*}
which means $ab\equiv a'b'$.
But then $a'b' - ab = -(ijn + a'j + b'i)n$ is also divisible by $n$, so $a'b'\equiv ab$.

Assume $ab\equiv a'b'$ and $a'b'\equiv cd$ which means that there exists integers $\tilde{i}$ and $\tilde{j}$ such that
\[
\begin{tabular}{ c c c c c }
$(ab) - (a'b') = \tilde{i}n$ & \hspace{1cm} & and & \hspace{1cm} & $(a'b') - (cd) = \tilde{j}n$.
\end{tabular}
\]
Again, we can deduce that
\[
ab - cd = (ab-a'b') + (a'b' - cd) = \tilde{i}n + \tilde{j}n = (\tilde{i} + \tilde{j})n
\]
which means $ab \equiv cd$.
\end{proof}

\section*{Revision}

\begin{proof}
Assume $a\equiv a'$ (mod $n$) and $b\equiv b'$ (mod $n$).  This can be expressed as follows:
\begin{align*}
n&|(a-a')\\
a-a'&=in\\
a&=in+a',
\end{align*}
and by following the same procedure, we can also write $b=jn+b'$ with $i,n\in\Z$.  Notice that
\begin{align*}
a+b &= (in +a') + (jn + b')\\
a+b &= (in + jn) + (a' + b')\\
(a+b) - (a' + b') &= (i+j)n,
\end{align*}
which means $a+b\equiv a' + b'$.

Similarly, we can observe that
\begin{align*}
ab &= a(jn + b')\\
ab &= ajn + ab'\\
ab &= (in + a')jn + (in + a')b'\\
ab &= injn + a'jn + b'in + a'b'\\
ab - a'b' &= (ijn + a'j + b'i)n,
\end{align*}
which means $ab\equiv a'b'$.
\end{proof}

\end{document}