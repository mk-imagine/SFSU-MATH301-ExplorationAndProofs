\documentclass[12pt]{amsart}
\usepackage[margin=1in]{geometry}

\newcommand{\Z}{\mathbb{Z}}
\newcommand{\N}{\mathbb{N}}

\newtheorem*{proposition}{Proposition}

\title{301 Homework 11}
\author{Mark Kim}
\date{\today}

\begin{document}
\maketitle

\begin{proposition}
Let $m\in\N$ and $n\in\Z$. If $mn\in\N$, then $n\in\N$.
\end{proposition}

\begin{proof}
Assume $m\in\N$, $n\in\Z$, and $mn\in\N$.  Proposition 2.2 states one and only one of the following applies: $n=0$, $-n\in\N$, $n\in\N$.  We can rule out the first case $n=0$ since $m\cdot0=0$ and Axiom 2.1(iii) explicitly states $0\notin\N$.  In the second case, $-n\in\N$, we have $m\cdot(-n)=-(mn)\in\N$, but this directly contradicts our assumption of $mn\in\N$.  Since the first two cases are not true, the last case must be true and $n\in\N$.  Thus we have proven that if $mn\in\N$, then $n\in\N$.
\end{proof}

\section*{Revision}

\begin{proof}
Assume $m\in\N$, $n\in\Z$, and $mn\in\N$.  Proposition 2.2 states one and only one of the following applies: $n=0$, $-n\in\N$, $n\in\N$.  We can rule out the first case $n=0$ since $m\cdot0=0$ and Axiom 2.1(iii) explicitly states $0\notin\N$.  In the second case, $-n\in\N$, we have $m\cdot(-n)=-(mn)\in\N$, but this contradicts Proposition 2.2 since we assume that $mn\in\N$.\footnote{In the first version of the proof, I did not justify the contradiction.  I simply cited Proposition 2.2 which states that only one case applies.} Since the first two cases are not true, the last case must be true and $n\in\N$.  Thus we have proven that if $mn\in\N$, then $n\in\N$.
\end{proof}

\end{document}