\documentclass[12pt]{amsart}
\usepackage[margin=1in]{geometry}
\usepackage{ amssymb }

\newcommand{\Z}{\mathbb{Z}}
\newcommand{\N}{\mathbb{N}}
\renewcommand{\complement}{^{\mathsf{c}}}

\newtheorem*{proposition}{Proposition}

\title{301 Homework 18}
\author{Mark Kim}
\date{\today}

\begin{document}
\maketitle

\begin{proposition}
Let $A,B\subseteq X$.
\begin{equation*}
\begin{tabular}{ c c c c c }
$A \subseteq B$ & & if and only if & & $B\complement \subseteq A\complement$\\
\end{tabular}
\end{equation*}
\end{proposition}

\begin{proof}
Assume $x\in A \Rightarrow x\in B$. Then, by contraposition, $x\notin B \Rightarrow x\notin A$.  Therefore $x\in B\complement \Rightarrow x\in A\complement$, which can be rewritten as $B\complement \subseteq A\complement$.

Similarly, to prove the other direction, we assume $x\in B\complement \Rightarrow x\in A\complement$.  The contrapositive, $x\notin A\complement \Rightarrow x\notin B\complement$, can be used to write $x\in A \Rightarrow x\in B$, which can be followed by $A\subseteq B$, proving our proposition.
\end{proof}

\end{document}