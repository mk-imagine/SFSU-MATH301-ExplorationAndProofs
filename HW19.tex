\documentclass[12pt]{amsart}
\usepackage[margin=1in]{geometry}
\usepackage{ amssymb }

\newcommand{\Z}{\mathbb{Z}}
\newcommand{\N}{\mathbb{N}}
\renewcommand{\complement}{^{\mathsf{c}}}

\newtheorem*{proposition}{Proposition}

\title{301 Homework 19}
\author{Mark Kim}
\date{\today}

\begin{document}
\maketitle

\begin{proposition}
Let $A,B,C$ be sets.
\[
A\times(B \cup C) = (A\times B)\cup(A\times C)
\]
\end{proposition}

\begin{proof}
Let $A,B,C$ be sets.  To prove our proposition, we need to show that
\[
A\times(B \cup C) \subseteq (A\times B)\cup(A\times C)
\]
and
\[
(A\times B)\cup(A\times C) \subseteq A\times(B \cup C).
\]

Let $x\in A\times(B \cup C)$, then $x=(y,z)$, where $y\in A$ and $z\in (B\cup C)$.  Evaluating $z$, it can be deduced that $z\in B$ or $z\in C$.  Since $y\in A$ for all cases, we can write the following:
\[
\begin{tabular}{c c c c c c c}
$y\in A \text{ and } z\in B$ & \text{    } & & or & & \text{    } & $y\in A \text{ and } z\in C$.
\end{tabular}
\]
This can be rewritten as $x\in A\times B$ or $x\in A\times C$. This proves that
\[
A\times(B \cup C) \subseteq (A\times B)\cup(A\times C).
\]

Conversely, if $x\in A \times B$ or $x\in A \times C$, where $x=(y,z)$, then
\[
\begin{tabular}{c c c c c c c}
$y\in A \text{ and } z\in B$ & \text{    } & & or & & \text{    } & $y\in A \text{ and } z\in C$.
\end{tabular}
\]
Because $y\in A$ in both cases and $z\in B$ or $z\in C$, we can now reason that $y\in A$ and $z\in (B\cup C)$.  Since $x\in A\times (B\cup C)$, we have proven
\[
(A\times B)\cup(A\times C) \subseteq A\times(B \cup C),
\]
which means
\[
A\times(B \cup C) = (A\times B)\cup(A\times C).
\]
\end{proof}

\end{document}