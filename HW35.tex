\documentclass[12pt]{amsart}
\usepackage[margin=1in]{geometry}
\usepackage{ amssymb }

\newcommand{\Z}{\mathbb{Z}}
\newcommand{\N}{\mathbb{N}}
\newcommand{\R}{\mathbb{R}}
\renewcommand{\complement}{^{\mathsf{c}}}

\newtheorem*{proposition}{Proposition}

\title{301 Homework 35}
\author{Mark Kim}
\date{\today}

\begin{document}
\maketitle

\begin{proposition}
If $x_1$ and $x_2$ are least upper bounds for $A$, then $x_1=x_2$.
\end{proposition}

\begin{proof}
Assume $x_1$ and $x_2$ are least upper bounds for $A$.  By definition, this means that $x_1$ is less than or equal to all other upper bounds of $A$ and since $x_2$ is also an upper bound of $A$, $x_1 \leq x_2$.  Likewise, $x_2$ is less than or equal to all other upper bounds of $A$ and since $x_1$ is also an upper bound of $A$, $x_2 \leq x_1$.  Since $x_1$ cannot be simultaneously less than and greater than $x_2$, $x_1=x_2$, proving the proposition.
\end{proof}
\end{document}