\documentclass[12pt]{amsart}
\usepackage[margin=1in]{geometry}

\newcommand{\Z}{\mathbb{Z}}
\newcommand{\N}{\mathbb{N}}

\newtheorem*{proposition}{Proposition}

\title{301 Homework 8}
\author{Mark Kim}
\date{\today}

\begin{document}
\maketitle

\begin{proposition}
$1\in\N$
\end{proposition}

\begin{proof}
With the goal of producing a contradiction, let us assume $1\notin\N$.  Axiom 2.1(iv) supplies us with three cases, of which $m\in\N$ does not apply.  We can also rule out $m=0$ since Axiom 1.3 explicitly states $1\neq0$.  Testing the third and final case $-m\in\N$, we can write $-1\in\N$.  Using Axiom 2.1(ii), the statement $mn\in\N$ can be written as $(-1)(-1)\in\N$ since $-1\in\N$.  Corollary 1.21 states $(-1)(-1)=1$, which gives us the result $1\in\N$, contradicting our initial assumption.  We have therefore proven that $1\in\N$.
\end{proof}

\end{document}