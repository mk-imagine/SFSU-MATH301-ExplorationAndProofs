\documentclass[12pt]{amsart}
\usepackage[margin=1in]{geometry}

\newcommand{\Z}{\mathbb{Z}}
\newcommand{\N}{\mathbb{N}}

\newtheorem*{proposition}{Proposition}

\title{301 Homework 14}
\author{Mark Kim}
\date{\today}

\begin{document}
\maketitle

\begin{proposition}
Let $A$ be a nonempty subset of $\Z$ and $b\in\Z$, such that for each $a\in A$, $b\leq a$. Then $A$ has a smallest element.
\end{proposition}

\begin{proof}
Assume $A$ is a nonempty subset of $\Z$ and $b\in\Z$ such that for each $a\in A$, $b\leq a$.  Furthermore, by definition, $b\leq a$ means $b=a$ or $b<a$ which can also be represented as
\[a-b=0\text{ or }a-b\in\N.\]
Notice that for our first case, if we add $1$ to both sides we get $a-b+1=1\in\N$ as stated by Proposition 2.3.  For our second case, we similary arrive at $a-b+1\in\N$ from using Axiom 2.1(i). Since both cases are true, we can form a nonempty set
\[ S:=\left\{ k\in\N:k = a-b+1\right\}.\]
Since $S\subseteq\N$, the Well-Ordering Principle affirms that $S$ has a smallest element.  I'm not completely sure if the rest here is correct, but I think that since we have a smallest element $k_0\in S$, we also have an element $a_0\in A$ such that $k_0 = a_0-b+1$.  Then we have to prove that for any element $a_x\in A$, $a_0\leq a_x$.  We can now state that $a_x-b+1\in S$, so it follows that
\begin{align*}
k_0&\leq a_x-b+1\\
a_0-b+1&\leq a_x-b+1.
\end{align*}
Writing the previous expression as \[(a_x-b+1)-(a_0-b+1)\in\N,\]allows us to simplify the statement and conclude that $a_x-a_0\in\N$ or $a_0\leq a_x$, proving the proposition.
\end{proof}

\end{document}
(a_x-b+1)-(a_0-b+1)&\in\N
a_x - a_0&\in\N\\
a_0&\leq a_x.