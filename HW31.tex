\documentclass[12pt]{amsart}
\usepackage[margin=1in]{geometry}
\usepackage{ amssymb }

\newcommand{\Z}{\mathbb{Z}}
\newcommand{\N}{\mathbb{N}}
\renewcommand{\complement}{^{\mathsf{c}}}

\newtheorem*{proposition}{Proposition}

\title{301 Homework 31}
\author{Mark Kim}
\date{\today}

\begin{document}
\maketitle

\begin{proposition}
The function $f\colon \N\rightarrow\Z$ defined by
\begin{align*}
f(n) =
\begin{cases}
\frac{1-n}{2},\text{\hspace{.1cm}when $n$ is odd}\\
\frac{n}{2},\text{\hspace{.5cm}when $n$ is even}
\end{cases}
\end{align*}
is bijective.
\end{proposition}

\begin{proof}
To prove this proposition, we must prove that the function $f\colon \N\rightarrow\Z$ defined by
\begin{align*}
f(n) =
\begin{cases}
\frac{1-n}{2},\text{\hspace{.1cm}when $n$ is odd}\\
\frac{n}{2},\text{\hspace{.5cm}when $n$ is even}
\end{cases}
\end{align*}
is both injective and surjective.

Let us begin by considering the contrapositive of the definition of injection $f(n_1)=f(n_2)\Rightarrow n_1=n_2$.  We have two cases to consider: when $n$ is odd and when $n$ is even.  Observe that when $n$ is odd
\begin{align*}
f(n_1) &= f(n_2)\\
\frac{1-n_1}{2} &= \frac{1-n_2}{2}\\
n_1 &= n_2.
\end{align*}
Similarly, when $n$ is even
\begin{align*}
f(n_1) &= f(n_2)\\
\frac{n_1}{2} &= \frac{n_2}{2}\\
n_1 &= n_2.
\end{align*}

I have no idea how to proceed with surjection.  I thought I would proceed by induction, but hit some brick walls.  I've tried just cooking up the inverse to the function, but I had no idea how to justify that.  So I'm at a complete loss here.
\end{proof}

\section{Revision}

\begin{proof}
To prove this proposition, we must prove that the function $f\colon \N\rightarrow\Z$ defined by
\begin{align*}
f(n) =
\begin{cases}
\frac{1-n}{2},\text{\hspace{.1cm}when $n$ is odd}\\
\frac{n}{2},\text{\hspace{.5cm}when $n$ is even}
\end{cases}
\end{align*}
is both injective and surjective.

We have three cases to consider when examining injectiveness: when $n_1$ is odd and $n_2$ is even, when both are even, and when both are odd.  Let us begin by considering the contrapositive of the definition of injection $f(n_1)=f(n_2)\Rightarrow n_1=n_2$.  Observe that when $n$ is odd
\begin{align*}
f(n_1) &= f(n_2)\\
\frac{1-n_1}{2} &= \frac{1-n_2}{2}\\
n_1 &= n_2.
\end{align*}
Similarly, when $n$ is even
\begin{align*}
f(n_1) &= f(n_2)\\
\frac{n_1}{2} &= \frac{n_2}{2}\\
n_1 &= n_2.
\end{align*}

Towards a contradiction, consider the case where $n_1$ is odd and $n_2$ is even and assume that $n_1 = n_2$.  Then 
\begin{align*}
f(n_1) &= f(n_2)\\
\frac{1-n_1}{2} &= \frac{n_2}{2}\\
1-n_1 &= n_2.
\end{align*}
Which means for any $n_1\in\N$, $n_2\notin \N$, but $n_2\in\N$.  Therefore it is never the case that $n_1$ is odd and $n_2$ is even.\footnote{Added this for completeness, since this should not be assumed.}

For surjectivity, let $m\in\Z$.  When $m\leq0$, $1-2m=n\in\N$, so
\begin{align*}
f(n) &= \frac{1-(1-2m)}{2}= \frac{2m}{2}=m.
\end{align*}
When $m>0$, $2m=n\in\N$, so
\begin{align*}
f(n) &= \frac{2m}{2} = m.
\end{align*}
Therefore for every $m\in\Z$ there exists $n\in\N$ such that $f(n)=m$.\footnote{Added surjection proof because I had no idea how to do it previously}
\end{proof}
\end{document}