\documentclass[12pt]{amsart}
\usepackage[margin=1in]{geometry}

\newcommand{\Z}{\mathbb{Z}}
\newcommand{\N}{\mathbb{N}}

\newtheorem*{proposition}{Proposition}

\title{301 Homework 16}
\author{Mark Kim}
\date{\today}

\begin{document}
\maketitle

\begin{proposition}
For all $k\in\N$, $2^{2k+1} + 1$ is divisible by 3.
\end{proposition}

\begin{proof}
Let $k\in\N$. We proceed by induction and must prove the base case and the induction step.

Using the base case, $k=1$, notice that $2^{2(1)+1} + 1=9$ is divisible by $3$.

Continuing with the induction step, we assume $2^{2n+1} + 1 = 3i$ for some $n\in\N$ and $i\in\Z$.  To start, we can simplify $2^{2(n+1)+1} + 1$,  to form $2^{2n+3} + 1.$ By applying Proposition 4.6(ii) and simplifying, we can see that
\begin{align*}
2^{2n+3} + 1 &= 2^{2n+1} \cdot 2 \cdot 2 + 1\\
&= 2^{2n+1} \cdot 4 + 1.
\end{align*}
Using our assumption, $2^{2n+1} + 1 = 3i$, we can surmise that $2^{2n+1}= 3i - 1$.  Substituting this back into $2^{2n+1} \cdot 4 + 1$, we can deduce that
\begin{align*}
	(3i-1)4 + 1 &= 12i - 4 + 1\\
	&= 12i-3\\
	&= 3(4i-1),
\end{align*}
which proves that for all $k\in\N$, $2^{2k+1} + 1$ is divisible by 3.
\end{proof}

\end{document}